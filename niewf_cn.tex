%!TEX program = xelatex
\documentclass[11pt,a4paper]{moderncv}

% moderncv themes
%\moderncvtheme[blue]{casual}                 % optional argument are 'blue' (default), 'orange', 'red', 'green', 'grey' and 'roman' (for roman fonts, instead of sans serif fonts)
\moderncvtheme[blue]{classic}                % idem
\usepackage{xunicode, xltxtra}
\XeTeXlinebreaklocale "zh"
\widowpenalty=10000

%\setmainfont[Mapping=tex-text]{文泉驿正黑}

% character encoding
%\usepackage[utf8]{inputenc}                   % replace by the encoding you are using
\usepackage{CJKutf8}
  
% adjust the page margins
\usepackage[scale=0.85]{geometry}
\recomputelengths                             % required when changes are made to page layout lengths
\setmainfont[Mapping=tex-text]{Hiragino Sans GB W3}
\setsansfont[Mapping=tex-text]{Hiragino Sans GB W3}
\CJKtilde

% personal data
\firstname{聂文峰}
\familyname{}
\title{}               % optional, remove the line if not wanted

\mobile{18515285263}                    % optional, remove the line if not wanted
\email{nwfengwolf@gmail.com}
% optional, remove the line if not wanted
\quote{\small{``Stay hungry, Stay foolish.''\\-- Steve Jobs}}

\nopagenumbers{}

\begin{document}

\maketitle

\section{专业技能}
\cventry{编程语言}{Python = Shell > Java > C++}{扎实的算法基础}{}{}{}
\cventry{机器学习}{熟悉word2vec、textcnn、rnn、Transformer以及Bert等常用NLP技术}{熟悉CF、LR、SVM等常用机器学习算法}{熟悉tensorflow框架}{}{}
\cventry{Hadoop}{熟练掌握MapReduce编程}{了解Hadoop运行机制及架构}{熟悉Hive}{了解HBase}{}
\cventry{版本管理}{熟练使用git和svn等版本管理工具}{}{}{}{}
\cventry{英语}{能用英语交流}{熟练阅读英文技术资料}{}{}{}

\closesection{}                   % needed to renewcommands
%\renewcommand{\listitemsymbol}{-} % change the symbol for lists

\section{工作经历}
\cventry{2016.03-- now}{杭州绿湾技术有限公司}{策略工程师}{}{}{大数据系统开发,NLP相关项目开发}
\cventry{2014.03--2016.03}{百度}{策略工程师}{}{}{新闻推荐系统的开发}
\cventry{2010.07--2014.03}{风行视频技术有限公司}{数据挖掘工程师}{}{}{数据仓库、推荐系统的开发}


%\section{社区}
%\cventry{Blog}{\url{nwf5d.github.com}}{技术博客}{}{}{}
%\cventry{StackOverflow}{\url{stackoverflow.com/users/111896/zellux}}{总排名前 4\%}{}{}{}
%\cventry{GitHub}{\url{github.com/nwf5d}}{参与过多个开源项目}{}{}{}

\section{项目经历}
\renewcommand{\baselinestretch}{1.2}

\vspace*{0.2\baselineskip}
\cventry{2019@绿湾}
{NLP--短文本分类}
{}
{}{}
{使用word2vec得到词向量,利用300篇人工标注数据,使用Bi-GRU+CRF实现提取人名、机构名、时间、POI等属性的识别.}

\cventry{2019}
{NLP--属性识别}
{}
{}{}
{使用word2vec得到词向量,利用300篇人工标注数据,使用Bi-GRU+CRF实现提取人名、机构名、时间、POI等属性的识别.}

\cventry{2018}
{NLP--中文信息抽取}
{}
{}{}
{通过查找资料,学习实践开发通用的中文信息抽取工具。主要命名实体识别、实体关系抽取和事件信息抽取三个功能.}

\cventry{2018}
{NLP--人物常驻地挖掘}
{}
{}{}
{从库中导出事件信息,抽取关注的身份证、时间、地点等字段,对人物、时间、地点、事件优化级、事件频率做统计,使用自举方法完成常驻地的挖掘.}

% \cventry{2018}
% {NLP--量刑模型预测}
% {}
% {}{}
% {采用内聚性、自由度,抽取聚簇的核心词对量刑做归一化处理,使用sklearn完成量刑回归模型训练.}

\cventry{2018}
{NLP--法院文书信息抽取}
{}
{}{}
{通过使用规则抽取法院判决书中的人、身份证号、被告/原告、时间等信息,用于知识图谱建模及后续挖掘.}

\cventry{2016-2018}
{大数据分析系统--分发层开发}
{知识图谱}
{Java}{}
{使用Spring Boot框架,完成分发层代码开发.}

\cventry{2015@百度}
{推荐系统--质量模型训练}
{内容模型}
{}{}
{人工标识一堆文本,评价分为高中低三档,分别训练高质量分类器(高/中低)和低质量分类器(高中/低),然后对每篇文章打上高质量分和低质量分,用于推荐排序及过滤。}
\cventry{2015}
{推荐系统--新词发现}
{内容模型}
{}{}
{通过人工标注色情新闻级别得到语料,使用词袋模型和LR模型建立一个分类模型,根据新闻标题和内容做色情度评定,用于解决CTR预估中色情新闻过多的问题}
\vspace*{0.2\baselineskip}
\cventry{2014}
{推荐系统--更新新闻分类器}
{内容模型}
{}{}
{通过人工标注新闻分类得到最近新闻分类语料,与之前的语料合并作为训练、测试与验证集,使用TFIDF计算权重,使用libsvm训练新闻分类器,解决因分类器太旧造成的分类错误。最终将分类准确率提升至93\%}

\vspace*{0.2\baselineskip}
\cventry{2014}
{推荐系统--新闻色情度分类}
{内容模型、评分打压过滤}
{}{}
{通过人工标注色情新闻级别得到语料,使用词袋模型和LR模型建立一个分类模型,根据新闻标题和内容做色情度评定,用于解决CTR预估中色情新闻过多的问题}

\vspace*{0.2\baselineskip}
\cventry{2013@风行}
{推荐系统--视频聚合影视推荐}
{关联推荐、冷启动}
{}{}
{通过抓取外网媒体信息(包括媒体简介、主演、导演、标签等)计算媒体间的相似度,给用户做关联推荐,用于解决冷启动问题。}

\vspace*{0.2\baselineskip}
\cventry{2013}
{推荐系统--LR模型对推荐结果二次排序}
{LR模型、二次排序}
{}{}
{采用用户反馈数据(隐式(观看、点击、浏览)+显式(顶、踩、打分等))进行LR建模,对CF的推荐结果进行二次排序,优化效果}

\vspace*{0.2\baselineskip}
\cventry{2013}{数据仓库--BIEE数据仓库迁移}{MapReduce、Kettle、BIEE}{}{}
{将数据仓库从原来的单机使用脚本及Infobright计算,迁移至Hadoop使用MapReduce,以提高系统的可扩展性和稳定性。同时建模和前端使用BIEE,数据库采用Oracle。数据连接及导入使用Kettle,通过数据分析与管理平台进行任务调度和管理。}

\vspace*{0.2\baselineskip}
\cventry{2012}
{数据仓库--数据分析与管理平台}
{Hadoop、Hive、PHP、MySQL、Python}
{}{}
{主要使用PHP+Python+MySQL实现一个统一管理风行所有上报数据、ETL数据以及调度任务的系统。目标是通过该系统,方便业务数据有管理、任务调度,并能开放整个Hadoop计算资源和数据资源给其他部门,降低各部门使用数据的门槛。}

\section{管理经历}
\renewcommand{\baselinestretch}{1.2}
\cventry{2012--2013}{Scrum敏捷开发}{数据BI团队的Scrum master,负责召开各种需求评审会、迭代会、回顾会等,带领团队完成BI组各类项目}{}{}{}
\cventry{2012--2013}{研发主管}{负责BI组的人员招聘、绩效考核、团队建设、人员培养等工作}{}{}{}
\renewcommand{\baselinestretch}{1.0}

%\renewcommand{\baselinestretch}{1.0}

\section{教育经历}
\cventry{2007.09--2010.07}{硕士}{北京工业大学}{计算机软件与理论}{}{}{}
\cventry{2003.10--2007.07}{本科}{南昌大学}{计算机科学与技术}{}{}{}  

\end{document}
