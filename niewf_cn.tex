\documentclass[11pt,a4paper]{moderncv}

% moderncv themes
%\moderncvtheme[blue]{casual}                 % optional argument are 'blue' (default), 'orange', 'red', 'green', 'grey' and 'roman' (for roman fonts, instead of sans serif fonts)
\moderncvtheme[blue]{classic}                % idem
\usepackage{xunicode, xltxtra}
\XeTeXlinebreaklocale "zh"
\widowpenalty=10000

%\setmainfont[Mapping=tex-text]{文泉驿正黑}

% character encoding
%\usepackage[utf8]{inputenc}                   % replace by the encoding you are using
\usepackage{CJKutf8}
  
% adjust the page margins
\usepackage[scale=0.8]{geometry}
\recomputelengths                             % required when changes are made to page layout lengths
\setmainfont[Mapping=tex-text]{Hiragino Sans GB W3}
\setsansfont[Mapping=tex-text]{Hiragino Sans GB W3}
\CJKtilde

% personal data
\firstname{聂文峰}
\familyname{}
\title{}               % optional, remove the line if not wanted

\mobile{18515285263}                    % optional, remove the line if not wanted
\email{nwfengwolf@gmail.com}                      % optional, remove the line if not wanted
\quote{\small{``Stay hungry, Stay foolish.''\\-- Steve Jobs}}

\nopagenumbers{}

\begin{document}

\maketitle

\section{专业技能}
\cventry{编程语言}{Python = Shell > Java > C++}{扎实的算法基础}{}{}{}
\cventry{机器学习}{熟悉推荐系统架构}{熟悉CF、SVD、LDA、pLSA、LR等常用机器学习算法}{}{}{}
\cventry{Hadoop}{熟练掌握MapReduce编程}{了解Hadoop运行机制及架构}{熟悉Hive}{了解HBase}{}
\cventry{数据仓库}{精通数据仓库建模过程}{熟悉使用Pentaho BI套件}{熟练掌握网站分析及数据分析的常见方法}{}{}
\cventry{英语}{能用英语交流}{熟练阅读英文技术资料}{}{}{}

\closesection{}                   % needed to renewcommands
%\renewcommand{\listitemsymbol}{-} % change the symbol for lists

\section{工作经历}
\cventry{2016-- now}{杭州绿湾技术有限公司}{策略工程师}{}{}{大数据系统开发}
\cventry{2014--2016}{百度}{策略工程师}{}{}{新闻推荐系统的开发}
\cventry{2010--2014}{风行视频技术有限公司}{数据挖掘工程师}{}{}{数据仓库、推荐系统的开发}


%\section{社区}
%\cventry{Blog}{\url{nwf5d.github.com}}{技术博客}{}{}{}
%\cventry{StackOverflow}{\url{stackoverflow.com/users/111896/zellux}}{总排名前 4\%}{}{}{}
%\cventry{GitHub}{\url{github.com/nwf5d}}{参与过多个开源项目}{}{}{}

\section{管理经历}
\renewcommand{\baselinestretch}{1.2}
\cventry{2012--2013}{Scrum敏捷开发}{数据BI团队的Scrum master,负责召开各种需求评审会、迭代会、回顾会等,带领团队完成BI组各类项目}{}{}{}
\cventry{2012--2013}{研发主管}{负责BI组的人员招聘、绩效考核、团队建设、人员培养等工作}{}{}{}
%\renewcommand{\baselinestretch}{1.0}

\section{项目经历}
\renewcommand{\baselinestretch}{1.2}

\cventry{2013~至今}
{推荐系统--视频聚合影视推荐}
{关联推荐、冷启动}
{}{}
{通过抓取外网媒体信息(包括媒体简介、主演、导演、标签等)计算媒体间的相似度,给用户做关联推荐,用于解决冷启动问题。}

\vspace*{0.2\baselineskip}
\cventry{2013}
{推荐系统--LR模型对推荐结果二次排序}
{LR模型、二次排序}
{}{}
{采用用户反馈数据(隐式(观看、点击、浏览)+显式(顶、踩、打分等))进行LR建模,对CF的推荐结果进行二次排序,优化效果}

\vspace*{0.2\baselineskip}
\cventry{2013}
{推荐系统--UGC视频聚类}
{C++、Python、Openmp(并行库)}
{}{}
{采用的pLSA聚类,该算法主要解决了并行化问题,之前基于LDA每次训练电影要花时间大约5-10个小时,现在速度提高到5-10分钟。pLSA主要用来是对小视频的剧情、描述、标题切词,然后训练每个小视频在主题上的分布,之后通过相似度计算聚类电影。}

\vspace*{0.2\baselineskip}
\cventry{2013}
{推荐系统--标签数据建模}
{标签数据、模型融合、Python}
{}{}
{通过抓取外部网站的标签数据,对媒体的标签数据进行建模,计算相似度,用于解决媒体冷启动问题。}

\vspace*{0.2\baselineskip}
\cventry{2013}{数据仓库--BIEE数据仓库迁移}{MapReduce、Kettle、BIEE}{}{}
{将数据仓库从原来的单机使用脚本及Infobright计算,迁移至Hadoop使用MapReduce,以提高系统的可扩展性和稳定性。同时建模和前端使用BIEE,数据库采用Oracle。数据连接及导入使用Kettle,通过数据分析与管理平台进行任务调度和管理。}

\vspace*{0.2\baselineskip}
\cventry{2012}
{数据仓库--数据分析与管理平台}
{Hadoop、Hive、PHP、MySQL、Python}
{}{}
{主要使用PHP+Python+MySQL实现一个统一管理风行所有上报数据、ETL数据以及调度任务的系统。目标是通过该系统,方便业务数据有管理、任务调度,并能开放整个Hadoop计算资源和数据资源给其他部门,降低各部门使用数据的门槛。}

%\renewcommand{\baselinestretch}{1.0}

\section{奖项}
\cventry{2012}{风行月度之星优秀员工}{}{}{}{}

\section{教育}
\cventry{2007--2010}{硕士}{北京工业大学}{计算机软件与理论}{}{}{}
\cventry{2003--2007}{本科}{南昌大学}{计算机科学与技术}{}{}{}  

\end{document}
